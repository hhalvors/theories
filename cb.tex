\documentclass[12pt]{article}
\usepackage{amsfonts}
\title{Cantor-Bernstein for Theories}
\author{Princeton Logic Group}
\usepackage{amsthm}
\usepackage[natbib=true,style=chicago-authordate,sorting=nyt]{biblatex}
% \usepackage{biblatex-chicago}
\addbibresource{/Users/hhalvors/org/ref/master.bib}
\newtheorem*{conj}{Conjecture}
\newtheorem*{prop}{Proposition}
\newtheorem*{fact}{Fact}
\theoremstyle{definition}
\newtheorem*{defn}{Definition}
\begin{document}

%% TO DO: Oligomorphic groups

%% TO DO: Ahlbrandt and Ziegler results

%% TO DO: put this on git-hub

\maketitle

The purpose of this note is to ask: under what conditions could a pair
of theories $(T_1,T_2)$ fail to have the Cantor-Bernstein or
co-Cantor-Bernstein property?

\begin{defn} We say that the pair $(T_1,T_2)$ has the Cantor-Bernstein
  property just in case: if $T_1$ and $T_2$ are mutually faithfully
  interpretable, then $T_1$ and $T_2$ are bi-interpretable. In other
  words: if $(T_1,T_2)$ does not have the Cantor-Bernstein property
  then (a) $T_1$ and $T_2$ are mutually faithfully interpretable, and
  (b) $T_1$ and $T_2$ are not bi-interpretable. \end{defn}

Here ``mutually faithfully interpretable'' means that there are
conservative (strong, equality preserving) translations $F:T_1\to T_2$
and $G:T_2\to T_1$.

\begin{defn} We say that the pair $(T_1,T_2)$ has the
  co-Cantor-Bernstein property just in case: if there are essentially
  surjective translations $F:T_1\to T_2$ and $G:T_2\to T_1$, then
  $T_1$ and $T_2$ are bi-interpretable. \end{defn}


\section{Examples of theory pairs that are not CB}


\begin{enumerate}
\item $T_1$ is the empty theory on a countably infinite propositional
  signature. $T_2$ is the ``fan theory'' with axioms $p_0\vdash p_i$,
  for $i\geq 0$.
\item The many examples of pairs of set theories described in \citep[p
  8]{hamkins}.
\item The pair of theories described in \citep{andreka}.
\end{enumerate}


\section{Results}

\begin{prop} If $T_1$ or $T_2$ is complete, then $(T_1,T_2)$ has the
  co-CB property. \end{prop}

\begin{proof} If $T_1$ is complete, then every translation
  $F:T_1\to T_2$ is conservative. So if $F:T_1\to T_2$ is eso, then
  $F$ is a strong equivalence, i.e.\ $T_1$ and $T_2$ are
  bi-interpretable. \end{proof}

Recall that $F:T_1\to T_2$ is conservative iff $F^*:M(T_2)\to M(T_1)$
is a full functor \citep{barrett2020}. Recall also that if $F$ is a
strong (equality-preserving) translation, then $F^*$ preserves
cardinality of models. In particular, if $T_1$ and $T_2$ are
$\aleph _0$-categorical theories, then a translation $F:T_1\to T_2$
induces a group homomorphism
$F^*:\mathrm{Aut}(M_2)\to \mathrm{Aut}(M_1)$, and $F$ is conservative
iff $F^*$ is surjective. (In fact, $\mathrm{Aut}(M_i)$ is naturally a
topological group, and I conjecture that $F^*$ is a continuous group
homomorphism.)

Recall that if $T$ has countable signature, and if $T$ is
$\aleph _0$-categorical, then $T$ is complete. The following result
would be interesting because an $\aleph _0$-categorical theory is
essentially characterized by a topological group, viz.\ the group of
automorphisms of its unique (up to isomorphism) countable model.


\begin{conj} There are $\aleph _0$-categorical theories $T_1$ and
  $T_2$ that do not have the CB property. \end{conj}

\section{Groups that are not CB}

\begin{defn} Let $P(\mathbb{N})$ be the permutation group of the
  natural numbers, equipped with the topology of pointwise
  convergence. (TO DO: explain the sense in which this topology on
  $P(\mathbb{N})$ is definable from the theory of infinite
  sets. Explain more generally the sense in which for a
  $\Sigma$-structure $M$, $\mathrm{Aut}(M)$ is naturally a topological
  group.) \end{defn}

\begin{fact} Let $G$ be a subgroup of $P(\mathbb{N})$. Then $G$ is the
  automorphism group of an $\aleph _0$-categorical theory iff $G$ is
  a closed subset of $P(\mathbb{N})$. \end{fact}

\begin{conj} There are closed subgroups $G$ and $H$ of
  $P(\mathbb{N})$ such that $G$ is isomorphic to a closed subgroup of
  $H$ and vice versa, but $G$ and $H$ are not
  isomorphic. \end{conj}

\begin{fact} There are closed subgroups $G$ and $H$ of $P(\mathbb{N})$
  that are isomorphic qua groups but not as topological
  groups. Accordingly, the corresponding theories are not
  bi-interpretable. \end{fact} 




\printbibliography 



\end{document} 
%%% Local Variables:
%%% mode: latex
%%% TeX-master: t
%%% End:
