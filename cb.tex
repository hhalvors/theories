\documentclass[12pt]{article}
\usepackage{doi}
\PassOptionsToPackage{hyphens}{url}
\usepackage{amsfonts,amsmath,extarrows,amssymb}
\title{Cantor-Bernstein for Theories}
\author{Princeton Logic Group}
\usepackage{amsthm}
\usepackage[natbib=true,style=chicago-authordate,sorting=nyt]{biblatex}
% \usepackage{biblatex-chicago}
\addbibresource{/Users/hhalvors/org/ref/master.bib}
\newtheorem{prop}{Proposition}
\newtheorem{fact}[prop]{Fact}
\newtheorem{cor}[prop]{Corollary}
\newtheorem{conj}[prop]{Conjecture}
\newtheorem{lemma}[prop]{Lemma}
\theoremstyle{definition}
\newtheorem*{defn}{Definition}
\newtheorem{example}[prop]{Example}
\theoremstyle{remark}
\newtheorem*{note}{Remark}

\newcommand{\3}{\mathcal}
\begin{document}

%% TO DO: Oligomorphic groups

%% TO DO: Ahlbrandt and Ziegler results

\maketitle

The purpose of this note is to ask: under what conditions could a pair
of theories $(T_1,T_2)$ fail to have the Cantor-Bernstein or
co-Cantor-Bernstein property?

\begin{defn} We say that the pair $(T_1,T_2)$ has the Cantor-Bernstein
  property just in case: if $T_1$ and $T_2$ are mutually faithfully
  interpretable, then $T_1$ and $T_2$ are bi-interpretable. In other
  words: if $(T_1,T_2)$ does not have the Cantor-Bernstein property
  then (a) $T_1$ and $T_2$ are mutually faithfully interpretable, and
  (b) $T_1$ and $T_2$ are not bi-interpretable. \end{defn}

Here ``mutually faithfully interpretable'' means that there are
conservative (strong, equality preserving) translations $F:T_1\to T_2$
and $G:T_2\to T_1$. It would also be interesting to look at this
question for the case of more general translations, with and without
equality preservation. Could we end up getting different kinds of
answers in the two cases? Or is the latter case reducible, in some
sense, to the former?

\begin{defn} We say that the pair $(T_1,T_2)$ has the
  co-Cantor-Bernstein property just in case: if there are essentially
  surjective translations $F:T_1\to T_2$ and $G:T_2\to T_1$, then
  $T_1$ and $T_2$ are bi-interpretable. \end{defn}

\newcommand{\2}{\mathsf}

\section{Factorization of translations}

For strong translations, we have a factorization theorem: a
translation $F$ is an equivalence iff $F$ is conservative and
essentially surjective \citep[Props.\ 4.5.26, 4.5.27]{tlps}. This fact
is useful for reasoning about relations between theories: if $T_1$ can
be embedded into $T_2$ such that every $\Sigma _2$-formula is
equivalent to a formula in the image, then $T_1$ and $T_2$ are
equivalent. It suggests the validity of the following mode of
reasoning about the relationship between theories:
\begin{quote}
  If $T_1$ can be interpreted into $T_2$ into such a way that (a) any
  inference licensed by $T_2$ was already licensed by $T_1$, and (b)
  any concept expressable by $T_2$ is already expressable by $T_1$,
  then $T_1$ and $T_2$ are equivalent. \end{quote} The purpose of this
section is to explore whether this kind of reasoning continues to hold
for generalized translations. The following example shows that it will
have to be modified in some way: $F$ being conservative and
essentially surjective does not imply that $F$ is an equivalence.

\begin{example} \label{nope} We show here that for a weak translations
  $F:T_1\to T_2$, conservative plus essentially surjective does not
  imply that $F$ is an equivalence. Let
  $\Sigma _1 = \{ \sigma _1,\sigma _2\}$, let
  $\Sigma _2= \{ \sigma \}$, and let $T_i$ be the empty theory in
  $\Sigma _i$. Define a reconstrual $F$ from $\Sigma _1$ to
  $\Sigma _2$ by setting $F(\sigma _i)=\sigma$, with trivial domain
  formula. Obviously $F:T_1\to T_2$ is a translation. We observe:
  \begin{itemize}
  \item $F$ is essentially surjective.
  \item $F$ is conservative.  
  \item $F$ is not part of an equivalence. Indeed, for any model $M$
    of $T_2$, the model $F^*(M)$ assigns the same set to $\sigma _1$
    and $\sigma _2$, hence $F^*$ is not essentially surjective.
  \end{itemize}
  The reason why $F$ is not an equivalence is because neither
  conservativity nor essential surjectivity require that the domain
  theory has a sufficient number of functional relations between
  sorts. In this case, $T_2$ has a functional relation between sorts
  $F(\sigma _1)$ and $F(\sigma _2)$ whereas $T_1$ does not have a
  functional relation between $\sigma _1$ and $\sigma _2$.  The issue
  then seems to be that $F$ does not resemble a full
  functor. Intuitively, a full functor would have the feature that for
  any functional relation $\chi :F(\sigma _1)\to F(\sigma _2)$, there
  is a functional relation $\theta :\sigma _1\to \sigma _2$ such that
  $F(\theta )\simeq \chi$. While there is indeed a $\Sigma _1$-formula
  $\eta$ such that $F(\eta )\simeq \chi$, any such $\eta$ has the
  wrong arity.

  In terms of the syntactic categories $C_{T_1}$ and $C_{T_2}$, we can
  see that the functor corresponding to $F$ is not full. In
  particular, if $x_i$ is a variable of sort $\sigma _i$, then there
  is an arrow from $[\top .Fx_1]$ to $[\top .Fx_2]$ in $C_{T_2}$, but
  no arrow from $[\top .x_1]$ to $[\top .x_2]$ in
  $C_{T_1}$.  \end{example}

\begin{example} Let
  $\Sigma _1 = \{ \sigma _1,\sigma _2,\chi _1,\chi _2\}$, where
  $\chi _i:\sigma _1\to \sigma _2$. Let $T_1$ be the theory in
  $\Sigma _1$ that says that $\chi _1$ and $\chi _2$ are functional
  relationships. Let $\Sigma _2 =\{ \sigma _1',\sigma _2',\theta \}$,
  and let $T_2$ be the theory in $\Sigma _2$ that says that $\theta$
  is a functional relationship. Let $F:\Sigma _1\to \Sigma _2$ be the
  reconstrual that takes both $\chi _1$ and $\chi _2$ to
  $\theta$. Then $F:T_1\to T_2$ is a translation, and as in the
  previous example, $F$ is essentially surjective. However, $F$ is not
  conservative. Indeed, we have
  $T_2\vdash F[\forall x\forall x(\chi _1\leftrightarrow \chi _2)]$
  but
  $T_1\not\vdash \forall x\forall y(\chi _1\leftrightarrow \chi _2)$.
\end{example}

Now here is my intuition: for a (weak) translation $F:T_1\to T_2$, the
notion of being conservative still makes sense --- although it may not
be a very powerful or useful notion. Indeed, to say that $F$ is
conservative basically says that if $F\phi (x)\vdash F\psi (x)$ then
$\phi (x)\vdash\psi (x)$, and that only tells us something about the
subobject lattices of the syntactic categories $C_{T_1}$ and
$C_{T_2}$. In contrast, to say that a logical functor from $C_{T_1}$
to $C_{T_2}$ is fully faithful means that for any fixed objects
$[\phi .X]$ and $[\psi .Y]$ of $C_{T_1}$, the map
\[ \mathrm{hom}\left( [\phi .X] ,[\psi .Y]\right) \: \to \:
  \mathrm{hom}\left( [F\phi .X] ,[F\psi .Y]\right) ,\] is a
bijection. The fullness condition implies that the translation is
conservative: if $F(\phi )\vdash F(\psi )$ in $T_2$, then $[X=X]$ is
an arrow from $F(\phi )$ to $F(\psi )$ in $C_{T_2}$. By the fullness
of $F$, there is a corresponding arrow from $\phi$ to $\psi$ in
$C_{T_1}$. Hence, $\phi\vdash\psi$ in $T_1$.

The next example shows that $F:T_1\to T_2$ being an equivalence does
not imply that $F$ is essentially surjective.

\begin{example} Let $T_1$ be the theory that says ``there are exactly
  two things,'' and let $T_2$ be the Morita extension of $T_1$ by the
  addition of a product sort. We know that the obvious embedding
  $F:T_1\to T_2$ is one-half of an equivalence. However, there are
  clearly formulas $\phi$ in the language of $T_2$ that are not
  provably equivalent to formulas in the target of the translation
  $F$. For example, let $\psi$ be the sentence that says there are
  four things of the product sort.

  Nonetheless, there is an extended sense in which this formula $\psi$
  is in the image of $F$. Indeed, if $\chi$ is the ``code'' for the
  product sort $\sigma '$, then for any $\Sigma _2$-formula $\psi (x)$
  with $x$ of type $\sigma '$, there is a $\Sigma _1$ formula
  $\phi (y)$ such that
  \[ T_2\: \vdash \: \chi (x,y)\to (\psi (x)\leftrightarrow \phi (x))
    .\] Notice that a code is really just another name for a t-map. I
  conjecture then that the correct formulation of ``essential
  surjectivity'' will necessarily involve t-maps. \end{example}

Let's now try to work at this in the other direction by looking at
abstract $2$-categoric notions of ``eso'' and ``fully faithful''
morphisms. See
\url{https://ncatlab.org/nlab/show/fully+faithful+morphism}

If $K$ is a 2-category, then for any objects $a,b$ of $K$, we let
$K(a,b)$ denote the category whose objects are $1$-cells (i.e. arrows)
from $a$ to $b$, and whose arrows are $2$-cells. Thus, in the
particular case of $\mathsf{Th}$, if $T_1$ and $T_2$ are theories,
then $\mathsf{Th}(T_1,T_2)$ is the category whose objects are
translations, and whose arrows are t-maps.

\begin{defn} In a $2$-category $\2K$, a $1$-cell $f:A\to B$ is said to
  be \emph{fully faithful} just in case for all objects $X\in \2K$,
  the induced functor $\2K (X,A)\to \2K (X,B)$ is full and
  faithful. \end{defn}

\begin{conj} Let $F:T_1\to T_2$ be a translation. Then $F$ is fully
  faithful iff $F$ is conservative. \end{conj}

NOTE: I now think this conjecture is false. I think that $F$ being
fully faithful is stronger than $F$ being conservative.




\section{Examples of theory pairs that are not CB}


\begin{enumerate}
\item $T_1$ is the empty theory on a countably infinite propositional
  signature. $T_2$ is the ``fan theory'' with axioms $p_0\vdash p_i$,
  for $i\geq 0$. These theories are counterexamples both to the CB and
  the co-CB properties.

  But are these theories ``pathological'' in some sense? The fact that
  these theories are propositional should not (I think) be seen as a
  pathology. However, these theories are incomplete, and the second of
  them is not finitely axiomatizable.
\item ZF and ZFC. See
  \url{https://cs.nyu.edu/pipermail/fom/2010-January/014325.html}
\item The many examples of pairs of set theories described in \citep[p
  8]{hamkins}.
\item The pair of theories described in \citep{andreka}.
\end{enumerate}

\section{Cantor-Bernstein}

The first thing I would like to show: for theories $T_1$ and $T_2$ to
violate Cantor-Bernstein, both have to have infinitely many
non-isomorphic models. (This fact is trivial in the case that $T_1$
and $T_2$ have models of infinite cardinality.)

\begin{note} Let $F:T_1\to T_2$ be a translation, and let
  $F^*:M(T_2)\to M(T_1)$ be the dual functor. Recall that:
  \begin{itemize}
  \item $F^*(M)\vDash \phi$ iff $M\vDash F(\phi )$, for any
    $\Sigma _1$-sentence $\phi$.
    \item $F^*$ is faithful by definition.
    \item $F^*$ reflects isomorphisms in the sense that if
      $F^*(j):F^*(M)\to F^*(N)$ is an isomorphism, then $j:M\to N$ is
      an isomorphism. This follows from the fact that an elementary
      embedding is an isomorphism iff it is surjective, and from the
      fact that the underlying function of $F^*(j)$ is none other than
      the underlying function of $j$.

      Note however: $F^*$ can still map non-isomorphic models to
      isomorphic models. For example: let $T_1$ be the theory that
      says that there are two things (in empty signature), let $T_2$
      be the theory that says that there are two things (in signature
      with a unary predicate symbol $P$), and let $F:T_1\to T_2$ be
      the inclusion. Let $M$ be a model in which the extension of $P$
      is empty, and let $N$ be a model in which the extension of $P$
      is non-empty. Then $F^*(M)$ is isomorphic to $F^*(N)$ although
      $M$ is not isomorphic to $N$.
    \end{itemize} \end{note}

\begin{defn} We say that a category $C$ is \emph{object-finite} just
  in case it has only finitely many objects up to
  isomorphism. \end{defn}

Let $T$ be a theory. If $M(T)$ is object-finite then the
L{\"o}wenheim-Skolem theorem implies that every model of $T$ has
finite cardinality. From this it follows that every hom set in $M(T)$
is finite. Furthermore, since for finite structures, elementary
equivalence implies isomorphism, it follows that for any
non-isomorphic $M,N$ in $M(T)$, there is a sentence $\phi$ such that
$M\vDash\phi$ and $N\vDash\neg\phi$.

\begin{prop} Let $T_1$ be a theory such that $M(T_1)$ is
  object-finite, and let $F:T_1\to T_2$ be a translation. If $F$ is
  conservative then $F^*:M(T_2)\to M(T_1)$ is essentially
  surjective. \end{prop}

\begin{proof} We prove the contrapositive. If $F^*:M(T_2)\to M(T_1)$
  is not eso, then there is a model $M$ of $T_1$ that is not
  isomorphic to any model of the form $F^*(N)$, with $N$ a model of
  $T_2$. By the preceding discussion, there is a sentence $\phi$ such
  that $M\vDash \phi$ but $F^*(N)\vDash\neg \phi$, for all models $N$
  of $T_2$. Hence $N\vDash F(\neg \phi )$ for all models $N$ of $T_2$,
  and by completeness, $T_2\vdash F(\neg \phi )$. Therefore $F$ is not
  conservative. \end{proof}

\begin{note} Let $C$ and $D$ be categories with respective object sets
  $C_0$ and $D_0$. Let $[C_0]$ and $[D_0]$ be the corresponding sets
  of equivalence classes of isomorphic objects. Each functor
  $F:C\to D$ induces a function $F_0:[C_0]\to [D_0]$, and $F_0$ is
  surjective iff $F$ is eso. If $F:C\to D$ and $G:D\to C$ are both
  eso, then Cantor-Bernstein for finite sets implies that $F_0$ is a
  bijection. \end{note}

\begin{lemma} Let $F$ be a finite set, let $f:F\to\mathbb{N}$ be a
  function, and let $\phi :F\to F$ be a bijection such that
  $f(x)\leq f(\phi (x))$ for all $x\in F$. Then $f(x)=f(\phi (x))$ for
  all $x\in F$. \end{lemma}

\begin{proof}[Sketch of proof] The function $f$ corresponds to a
  fibration of $F$ over $\mathbb{N}$. Since $\phi$ is a bijection, the
  size of the fibers remain constant, i.e.,
  $|f^{-1}(n)|=|(f\circ \phi )^{-1}(n)|$. Since $\phi$ is monotonic,
  it cannot move an element to a lower fiber. Thus no element can be
  moved out of the highest fiber, nor the next highest fiber,
  etc. \end{proof}

\begin{prop} Let $C$ and $D$ be totally-finite categories. If there
  are faithful, eso functors $F:C\to D$ and $G:D\to C$, then $C$ and
  $D$ are equivalent categories. In fact, $F$ itself is one half of an
  equivalence. \end{prop}

\begin{proof} By the above remark, $F_0:[C_0]\to [D_0]$ is a
  bijection. Since $F$ is automatically faithful, it will suffice to
  show that $F$ is full. For simplicity, we may henceforth replace $C$
  and $D$ with the corresponding skeletal categories.

  Consider the (finite) set $C_0\times C_0$ and the function
  $f:C_0\times C_0\to \mathbb{N}$ that assigns the cardinality of the
  corresponding hom set. That is, $f(a,b)=|\mathrm{hom}(a,b)|$. Let
  $g:D_0\times D_0\to\mathbb{N}$ be the corresponding function for
  $D$. Since $F$ is faithful, it induces a bijection
  $\eta :C_0\to D_0$ such that
  \[ f(a,b) \: \leq \: g(\eta (a),\eta (b)) .\] And since $G$ is
  faithful, it induces a bijection $\theta :D_0\to C_0$ such that
  \[ g(a,b) \: \leq \: f(\theta (a),\theta (b)) .\] If we let
  $\phi = \theta \circ\eta$ then
  \[ f(a,b) \: \leq \: f(\phi (a),\phi (b)) ,\] for all $a,b\in
  C_0$. By the above lemma, $f(a,b)=f(\phi (a),\phi (b))$, and it
  follows that $F$ is full. \end{proof}

\begin{conj} Suppose that $F:T_1\to T_2$ is a strong translation,
  i.e.\ one-dimensional, equality preserving, and with trivial domain
  formula. If $F$ is one-half of a weak equivalence, then $F$ is
  one-half of a strong equivalence. \end{conj}

\begin{proof}[Sketch of proof] We show that $F$ is conservative and
  essentially surjective. Let $G:T_2\to T_1$ be a translation such
  that $GF\simeq 1_{T_1}$ and $FG\simeq 1_{T_2}$. Let $\phi$ be a
  $\Sigma _1$-sentence such that $T_2\vdash F(\phi )$. Then
  $T_1\vdash GF(\phi )$, and (since the relevant t-map is trivial)
  $T_1\vdash \phi$. Therefore $F$ is conservative. Now let $\psi (x)$
  be a $\Sigma _2$-formula, for simplicity with one free
  variable. Then there a t-map $\chi (x,y)$ such that
  \[ T_2\: \vdash \: \chi (x,y)\to (\psi (x)\leftrightarrow (FG\psi
    )(y)) ,
  \] and $T_2$ implies that $\chi$ is a bijection on the domain. TO BE CONTINUED
 
\end{proof}

\begin{prop} Let $T_1$ and $T_2$ be proper theories such that $M(T_1)$
  and $M(T_2)$ are object-finite. Then $(T_1,T_2)$ has the
  Cantor-Bernstein property. \end{prop}

\begin{proof} Let $F:T_1\to T_2$ and $G:T_2\to T_1$ be conservative
  translations. By the previous results, $F^*:M(T_2)\to M(T_1)$ is
  part of an equivalence of categories. By Theorem 7.1 of
  \citep{summer2020}, $F$ is part of a homotopy
  equivalence.  \end{proof}

\begin{note} TO DO: I need to check the previous result. The problem
  is that the conclusion of Theorem 7.1 shows that $F$ is part of a
  weak equivalence. Does that automatically show that $F$ is part of a
  strong equivalence? Note that the dimension of $F$ is $1$, and the
  domain formula is trivial/universal. \end{note}

\begin{conj} If $M(T_1)$ and $M(T_2)$ are object-finite then
  $(T_1,T_2)$ has the Cantor-Bernstein property.  \end{conj}

\section{Co-Cantor-Bernstein}


\begin{prop} If $T_1$ or $T_2$ is complete, then $(T_1,T_2)$ has the
  co-CB property. \end{prop}

\begin{proof} If $T_1$ is complete, then every translation
  $F:T_1\to T_2$ is conservative. So if $F:T_1\to T_2$ is eso, then
  $F$ is a strong equivalence, i.e.\ $T_1$ and $T_2$ are
  bi-interpretable. \end{proof}

Recall that $F:T_1\to T_2$ is conservative iff $F^*:M(T_2)\to M(T_1)$
is a full functor \citep{barrett2020}. Recall also that if $F$ is a
strong (equality-preserving) translation, then $F^*$ preserves
cardinality of models. In particular, if $T_1$ and $T_2$ are
$\aleph _0$-categorical theories, then a translation $F:T_1\to T_2$
induces a group homomorphism
$F^*:\mathrm{Aut}(M_2)\to \mathrm{Aut}(M_1)$, and $F$ is conservative
iff $F^*$ is surjective. (In fact, $\mathrm{Aut}(M_i)$ is naturally a
topological group, and I conjecture that $F^*$ is a continuous group
homomorphism.)

Recall that if $T$ has countable signature, and if $T$ is
$\aleph _0$-categorical, then $T$ is complete. The following result
would be interesting because an $\aleph _0$-categorical theory is
essentially characterized by a topological group, viz.\ the group of
automorphisms of its unique (up to isomorphism) countable model.


\begin{conj} There are $\aleph _0$-categorical theories $T_1$ and
  $T_2$ that do not have the CB property. \end{conj}

\begin{defn} Let $I$ be the set of axioms
  $\{ \exists _{>1},\exists _{>2},\dots \}$. We say that a theory $T$
  is essentially finitely axiomatizable just in case there is a finite
  set $E$ of axioms such that $Cn(T)=Cn(E\cup I)$. \end{defn}

\begin{conj} There are essentially finitely axiomatizable theories
  $T_1$ and $T_2$ that do not have the CB property. \end{conj}

\section{Conjecture: co-CB fails only for theories with many models}

\begin{defn} For a theory $T$ and a cardinal number $\kappa$, let
  $I(T,\kappa )$ be the number of non-isomorphic countable models of
  $T$. \end{defn}

Interestingly, $I(T,\aleph _0)$ can be countably infinite, or any
finite cardinal besides $2$ \citep[see][p 155ff]{marker2006}.

\begin{prop} If $F:T_1\to T_2$ is essentially surjective, then for any
  fixed cardinal number $\kappa$,
  $I(T_2, \kappa )\leq I(T_1, \kappa )$. \end{prop}

\begin{proof} Recall that the dual functor $F^*$ is always
  faithful. If $F:T_1\to T_2$ is eso, then $F^*$ is also full
  \citep[Prop 6.6.13]{tlps}. In particular, for any models $M,N$ of
  $T_2$, if $F^*(M)$ is isomorphic to $F^*(N)$, then $M$ is isomorphic
  to $N$. Now fix a cardinal number $\kappa$, and let
  $[M(T_i)]_\kappa$ be the set of isomorphism classes of models of
  $T_i$ of cardinality $\kappa$. Then $F^*$ induces a one-to-one
  mapping from $[M(T_2)]_\kappa$ into $[M(T_1)]_\kappa$. \end{proof}

\begin{prop} Suppose that $T_2$ has finitely many non-isomorphic
  models of each cardinality. If $F:T_1\to T_2$ and $G:T_2\to T_1$ are
  essentially surjective, then $(F^*,G^*)$ is an equivalence of
  categories. If $T_1$ and $T_2$ are proper theories then $F^*$ is
  part of a homotopy equivalence. \end{prop}

\begin{proof} For the first part it will suffice to show that $F^*$ is
  essentially surjective. By the previous proof, $G^*$ induces an
  injection of $[M(T_1)]_\kappa$ into $[M(T_2)]_\kappa$. Since the
  latter is finite, so is the former. Since $F^*$ is an injection of
  one finite set into a not-larger finite set, it follows that $F^*$
  is bijection. Therefore $F^*$ is essentially surjective.

  The second part follows from Theorem 7.1 of
  \citep{summer2020}. \end{proof}

\begin{cor} Let $T_1$ and $T_2$ be proper theories. If $(T_1,T_2)$
  violate the co-CB property then there is a cardinal number $\kappa$
  such that $T_1$ and $T_2$ have infinitely many non-isomorphic models
  of size $\kappa$. \end{cor}

TO DO: I would like to come up with a similar necessary condition for
$T_1$ and $T_2$ violating the CB property. However, we do not yet have
any interesting result of the form: ``if $T_1$ or $T_2$ is \dots and
$F:T_1\to T_2$ is conservative then $F^*:M(T_2)\to M(T_1)$ is \dots
.''



\section{Groups that are not CB}

\begin{defn} Let $P(\mathbb{N})$ be the permutation group of the
  natural numbers, equipped with the topology of pointwise
  convergence. (TO DO: explain the sense in which this topology on
  $P(\mathbb{N})$ is definable from the theory of infinite
  sets. Explain more generally the sense in which for a
  $\Sigma$-structure $M$, $\mathrm{Aut}(M)$ is naturally a topological
  group.) \end{defn}

\begin{fact} Let $G$ be a subgroup of $P(\mathbb{N})$. Then $G$ is the
  automorphism group of an $\aleph _0$-categorical theory iff $G$ is
  a closed subset of $P(\mathbb{N})$. \end{fact}

\begin{conj} There are closed subgroups $G$ and $H$ of
  $P(\mathbb{N})$ such that $G$ is isomorphic to a closed subgroup of
  $H$ and vice versa, but $G$ and $H$ are not
  isomorphic. \end{conj}

It is not difficult at all to find groups that violate the
Cantor-Bernstein condition --- but I do not immediately know if any of
these groups are of the form $\mathrm{Aut}(M)$ for an
$\aleph _0$-categorical structure $M$.

\begin{enumerate}
\item The group $S_\infty$ of finite permutations of $\mathbb{N}$ and
  the alternating grooup $A_\infty$. See  \url{https://math.stackexchange.com/questions/1259081/if-there-are-injective-homomorphisms-between-two-groups-in-both-directions-are}
\item Infinite direct sums of $\mathbb{Z}_{2^i}$.
\item The free group on $2$ generators and the free group on $3$ generators.  
\end{enumerate}

\begin{prop}[\cite{ahlbrandt1986}] Two countable
  $\aleph _0$-categorical structures are bi-interpretable iff their
  automorphism groups are isomorphic as topological groups. \end{prop}

Conjecture: the previous result can be lifted to
$\aleph _0$-categorical theories (with countable signature). But we
need to be careful about terminology. First of all, Ahlbrandt and
Ziegler are working with a notion of ``interpretation'' between
structures of one language and structures of another language: given a
$\Sigma _1$-structure $\3M _1$ and a $\Sigma _2$-structure $\3M _2$,
and interpretation $f:\3M_1\to \3M_2$ consists of a surjection
$f:U\to M_2$ where $U$ is a definable subset of the domain of $\3M_1$
and $M_2$ is the domain of $\3M_2$, etc.

\begin{conj} An interpretation from $\3M_1$ to $\3M_2$ is a
  translation in our sense from $Th(\3M_1)$ to $Th(\3M_2)$. \end{conj}

The following proposition follows immediately from the fact that
$\aleph _0$-categorical theories are complete.

\begin{prop} If $T$ is $\aleph _0$-categorical with unique model
  $\3M$, then $T$ is logically equivalent to $Th(\3M )$. \end{prop}

This result would be interesting because it would show that the
structure of an $\aleph _0$-categorical theory is captured by its
countable model. i.e., there is no need to look at the models of
higher cardinality and the arrows between them.

\begin{prop}[\cite{evans1990}] There are closed subgroups $G$ and $H$
  of $P(\mathbb{N})$ that are isomorphic qua groups but not as
  topological groups. Hence, the corresponding structures are not
  bi-interpretable. \end{prop}

The former result is intriguing: there is a group isomorphism
$\varphi :G\xlongrightarrow{\sim} H$ that does not correspond to a
homotopy equivalence between the corresponding theories. In short: the
symmetries of the model are not enough to capture the structure of the
theories.

\section{$\aleph _0$-categorical theories}

For this discussion, we restrict to theories with countable
signatures. In this case, the downward L{\"o}wenheim-Skolem theorem
shows that an $\aleph _0$-categorical theory is complete.

Question: How can we characterize the syntactic categories of
$\aleph _0$-categorical theories? (Hint: Look at the Ryll-Nardzewski
theorem \url{https://en.wikipedia.org/wiki/Omega-categorical_theory}
and \citep[p 30]{cameron1990}. I suspect that the characterization
will have something to do with finiteness of subobject lattices.)

TO DO: Establish a correspondence between the $2$-category of
$\aleph _0$-categorical theories and some subcategory of the
$2$-category of topological groups. Note 1: every group is a category,
and group homomorphisms are functors.  Note 2: all translations
between such theories are conservative. It will be helpful to look at
pages 106--107 of \citep{cameron1990} where he describes conditions on
a topological group $G$ that ensure it corresponds to a categorical
theory.



\begin{defn} Given an $\aleph _0$-categorical theory $T$, let
  $\3G (T)$ be the topological group of automorphisms of its unique
  countable model. \end{defn}

\begin{conj} If $F:T_1\to T_2$ is a translation then
  $F^*|_{\3G (T_2)}$ is a continuous homomorphism from $\3G (T_2 )$ to
  $\3G (T_1)$. \end{conj}

\begin{conj} There is a 2-functor $\3G$ from the 2-category of
  topological groups to the 2-category of $\aleph _0$-categorical
  theories. (This is not stated correctly yet: we should not expect
  all topological groups to occur in the domain. It is only the
  ``nice'' topological groups, i.e.\ the automorphism groups of
  $\aleph _0$-categorical theories.) \end{conj}


\begin{conj} There is a 2-functor $\3F$ from the 2-category of
  $\aleph _0$-categorical theories and the 2-category of topological
  groups. \end{conj}







\printbibliography 



\end{document} 
%%% Local Variables:
%%% mode: latex
%%% TeX-master: t
%%% End:
