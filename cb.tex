\documentclass[12pt]{article}
\usepackage{doi}
\PassOptionsToPackage{hyphens}{url}
\usepackage{amsfonts,amsmath,extarrows}
\title{Cantor-Bernstein for Theories}
\author{Princeton Logic Group}
\usepackage{amsthm}
\usepackage[natbib=true,style=chicago-authordate,sorting=nyt]{biblatex}
% \usepackage{biblatex-chicago}
\addbibresource{/Users/hhalvors/org/ref/master.bib}
\newtheorem*{conj}{Conjecture}
\newtheorem*{prop}{Proposition}
\newtheorem*{fact}{Fact}
\newtheorem*{cor}{Corollary}
\theoremstyle{definition}
\newtheorem*{defn}{Definition}
\newcommand{\3}{\mathcal}
\begin{document}

%% TO DO: Oligomorphic groups

%% TO DO: Ahlbrandt and Ziegler results

\maketitle

The purpose of this note is to ask: under what conditions could a pair
of theories $(T_1,T_2)$ fail to have the Cantor-Bernstein or
co-Cantor-Bernstein property?

\begin{defn} We say that the pair $(T_1,T_2)$ has the Cantor-Bernstein
  property just in case: if $T_1$ and $T_2$ are mutually faithfully
  interpretable, then $T_1$ and $T_2$ are bi-interpretable. In other
  words: if $(T_1,T_2)$ does not have the Cantor-Bernstein property
  then (a) $T_1$ and $T_2$ are mutually faithfully interpretable, and
  (b) $T_1$ and $T_2$ are not bi-interpretable. \end{defn}

Here ``mutually faithfully interpretable'' means that there are
conservative (strong, equality preserving) translations $F:T_1\to T_2$
and $G:T_2\to T_1$.

\begin{defn} We say that the pair $(T_1,T_2)$ has the
  co-Cantor-Bernstein property just in case: if there are essentially
  surjective translations $F:T_1\to T_2$ and $G:T_2\to T_1$, then
  $T_1$ and $T_2$ are bi-interpretable. \end{defn}




\section{Examples of theory pairs that are not CB}


\begin{enumerate}
\item $T_1$ is the empty theory on a countably infinite propositional
  signature. $T_2$ is the ``fan theory'' with axioms $p_0\vdash p_i$,
  for $i\geq 0$. These theories are counterexamples both to the CB and
  the co-CB properties.

  But are these theories ``pathological'' in some sense? The fact that
  these theories are propositional should not (I think) be seen as a
  pathology. However, these theories are incomplete, and the second of
  them is not finitely axiomatizable.
\item The many examples of pairs of set theories described in \citep[p
  8]{hamkins}.
\item The pair of theories described in \citep{andreka}.
\end{enumerate}


\section{Results}

\begin{prop} If $T_1$ or $T_2$ is complete, then $(T_1,T_2)$ has the
  co-CB property. \end{prop}

\begin{proof} If $T_1$ is complete, then every translation
  $F:T_1\to T_2$ is conservative. So if $F:T_1\to T_2$ is eso, then
  $F$ is a strong equivalence, i.e.\ $T_1$ and $T_2$ are
  bi-interpretable. \end{proof}

Recall that $F:T_1\to T_2$ is conservative iff $F^*:M(T_2)\to M(T_1)$
is a full functor \citep{barrett2020}. Recall also that if $F$ is a
strong (equality-preserving) translation, then $F^*$ preserves
cardinality of models. In particular, if $T_1$ and $T_2$ are
$\aleph _0$-categorical theories, then a translation $F:T_1\to T_2$
induces a group homomorphism
$F^*:\mathrm{Aut}(M_2)\to \mathrm{Aut}(M_1)$, and $F$ is conservative
iff $F^*$ is surjective. (In fact, $\mathrm{Aut}(M_i)$ is naturally a
topological group, and I conjecture that $F^*$ is a continuous group
homomorphism.)

Recall that if $T$ has countable signature, and if $T$ is
$\aleph _0$-categorical, then $T$ is complete. The following result
would be interesting because an $\aleph _0$-categorical theory is
essentially characterized by a topological group, viz.\ the group of
automorphisms of its unique (up to isomorphism) countable model.


\begin{conj} There are $\aleph _0$-categorical theories $T_1$ and
  $T_2$ that do not have the CB property. \end{conj}

\begin{defn} Let $I$ be the set of axioms
  $\{ \exists _{>1},\exists _{>2},\dots \}$. We say that a theory $T$
  is essentially finitely axiomatizable just in case there is a finite
  set $E$ of axioms such that $Cn(T)=Cn(E\cup I)$. \end{defn}

\begin{conj} There are essentially finitely axiomatizable theories
  $T_1$ and $T_2$ that do not have the CB property. \end{conj}

\section{Conjecture: co-CB fails only for theories with many models}

\begin{prop} If $F:T_1\to T_2$ is essentially surjective, then for any
  fixed cardinal number $\kappa$, $T_1$ has more models (up to
  isomorphism) of size $\kappa$ than~$T_2$. \end{prop}

\begin{proof} Recall that the dual functor $F^*$ is always
  faithful. If $F:T_1\to T_2$ is eso, then $F^*$ is also full
  \citep[Prop 6.6.13]{tlps}. In particular, for any models $M,N$ of
  $T_2$, if $F^*(M)$ is isomorphic to $F^*(N)$, then $M$ is isomorphic
  to $N$. Now fix a cardinal number $\kappa$, and let
  $[M(T_i)]_\kappa$ be the set of isomorphism classes of models of
  $T_i$ of cardinality $\kappa$. Then $F^*$ induces a one-to-one
  mapping from $[M(T_2)]_\kappa$ into $[M(T_1)]_\kappa$. \end{proof}

\begin{prop} Suppose that $T_2$ has finitely many non-isomorphic
  models of each cardinality. If $F:T_1\to T_2$ and $G:T_2\to T_1$ are
  essentially surjective, then $(F^*,G^*)$ is an equivalence of
  categories. \end{prop}

\begin{proof} It will suffice to show that $F^*$ is essentially
  surjective. By the previous proof, $G^*$ induces an injection of
  $[M(T_1)]_\kappa$ into $[M(T_2)]_\kappa$. Since the latter is
  finite, so is the former. Since $F^*$ is an injection of one finite
  set into a not-larger finite set, it follows that $F^*$ is
  bijection. Therefore $F^*$ is essentially surjective. \end{proof}

Having $(F^*,G^*)$ an equivalence of categories is tantalizingly close
to implying that $(F,G)$ is a homotopy equivalence. In fact,
Makkai-Reyes conceptual completeness shows (I think?) that if $T_1$
and $T_2$ Morita complete, then $F^*$ is a categorical equivalence
only if $F$ is a homotopy equivalence.

\begin{conj} If $(T_1,T_2)$ violate the co-CB property then there is a
  cardinal number $\kappa$ such that $T_1$ and $T_2$ have infinitely
  many non-isomorphic models of size $\kappa$. \end{conj}

I suspect that this conjecture follows from the previous
proposition. My intuition here is that when theories have finitely
many non-isomorphic models (of each fixed cardinality), then
categorical equivalence implies bi-interpretability.

\begin{conj} Suppose that $T_1$ and $T_2$ have finitely many
  non-isomorphic models of each fixed cardinal number. If
  $F:T_1\to T_2$ and $G:T_2\to T_1$ are translations such that
  $(F^*,G^*)$ is an equivalence of categories, then $(F,G)$ is a
  homotopy equivalence. \end{conj}


TO DO: I would like to come up with a similar necessary condition for
$T_1$ and $T_2$ violating the CB property. However, we do not yet have
any interesting result of the form: ``if $T_1$ or $T_2$ is \dots and
$F:T_1\to T_2$ is conservative then $F^*:M(T_2)\to M(T_1)$ is \dots
.''



\section{Groups that are not CB}

\begin{defn} Let $P(\mathbb{N})$ be the permutation group of the
  natural numbers, equipped with the topology of pointwise
  convergence. (TO DO: explain the sense in which this topology on
  $P(\mathbb{N})$ is definable from the theory of infinite
  sets. Explain more generally the sense in which for a
  $\Sigma$-structure $M$, $\mathrm{Aut}(M)$ is naturally a topological
  group.) \end{defn}

\begin{fact} Let $G$ be a subgroup of $P(\mathbb{N})$. Then $G$ is the
  automorphism group of an $\aleph _0$-categorical theory iff $G$ is
  a closed subset of $P(\mathbb{N})$. \end{fact}

\begin{conj} There are closed subgroups $G$ and $H$ of
  $P(\mathbb{N})$ such that $G$ is isomorphic to a closed subgroup of
  $H$ and vice versa, but $G$ and $H$ are not
  isomorphic. \end{conj}

It is not difficult at all to find groups that violate the
Cantor-Bernstein condition --- but I do not immediately know if any of
these groups are of the form $\mathrm{Aut}(M)$ for an
$\aleph _0$-categorical structure $M$.

\begin{enumerate}
\item The group $S_\infty$ of finite permutations of $\mathbb{N}$ and
  the alternating grooup $A_\infty$. See  \url{https://math.stackexchange.com/questions/1259081/if-there-are-injective-homomorphisms-between-two-groups-in-both-directions-are}
\item Infinite direct sums of $\mathbb{Z}_{2^i}$.
\item The free group on $2$ generators and the free group on $3$ generators.  
\end{enumerate}

\begin{prop}[\cite{ahlbrandt1986}] Two countable
  $\aleph _0$-categorical structures are bi-interpretable iff their
  automorphism groups are isomorphic as topological groups. \end{prop}

Conjecture: the previous result can be lifted to
$\aleph _0$-categorical theories (with countable signature). But we
need to be careful about terminology. First of all, Ahlbrandt and
Ziegler are working with a notion of ``interpretation'' between
structures of one language and structures of another language: given a
$\Sigma _1$-structure $\3M _1$ and a $\Sigma _2$-structure $\3M _2$,
and interpretation $f:\3M_1\to \3M_2$ consists of a surjection
$f:U\to M_2$ where $U$ is a definable subset of the domain of $\3M_1$
and $M_2$ is the domain of $\3M_2$, etc.

\begin{conj} An interpretation from $\3M_1$ to $\3M_2$ is a
  translation in our sense from $Th(\3M_1)$ to $Th(\3M_2)$. \end{conj}

The following proposition follows immediately from the fact that
$\aleph _0$-categorical theories are complete.

\begin{prop} If $T$ is $\aleph _0$-categorical with unique model
  $\3M$, then $T$ is logically equivalent to $Th(\3M )$. \end{prop}

This result would be interesting because it would show that the
structure of an $\aleph _0$-categorical theory is captured by its
countable model. i.e., there is no need to look at the models of
higher cardinality and the arrows between them.

\begin{prop}[\cite{evans1990}] There are closed subgroups $G$ and $H$
  of $P(\mathbb{N})$ that are isomorphic qua groups but not as
  topological groups. Hence, the corresponding structures are not
  bi-interpretable. \end{prop}

The former result is intriguing: there is a group isomorphism
$\varphi :G\xlongrightarrow{\sim} H$ that does not correspond to a
homotopy equivalence between the corresponding theories. In short: the
symmetries of the model are not enough to capture the structure of the
theories.

\section{$\aleph _0$-categorical theories}

For this discussion, we restrict to theories with countable
signatures. In this case, the downward L{\"o}wenheim-Skolem theorem
shows that an $\aleph _0$-categorical theory is complete.

Question: How can we characterize the syntactic categories of
$\aleph _0$-categorical theories? (Hint: Look at the Ryll-Nardzewski
theorem \url{https://en.wikipedia.org/wiki/Omega-categorical_theory}
and \citep[p 30]{cameron1990}. I suspect that the characterization
will have something to do with finiteness of subobject lattices.)

TO DO: Establish a correspondence between the $2$-category of
$\aleph _0$-categorical theories and some subcategory of the
$2$-category of topological groups. Note 1: every group is a category,
and group homomorphisms are functors.  Note 2: all translations
between such theories are conservative. It will be helpful to look at
pages 106--107 of \citep{cameron1990} where he describes conditions on
a topological group $G$ that ensure it corresponds to a categorical
theory.



\begin{defn} Given an $\aleph _0$-categorical theory $T$, let
  $\3G (T)$ be the topological group of automorphisms of its unique
  countable model. \end{defn}

\begin{conj} If $F:T_1\to T_2$ is a translation then
  $F^*|_{\3G (T_2)}$ is a continuous homomorphism from $\3G (T_2 )$ to
  $\3G (T_1)$. \end{conj}

\begin{conj} There is a 2-functor $\3G$ from the 2-category of
  topological groups to the 2-category of $\aleph _0$-categorical
  theories. (This is not stated correctly yet: we should not expect
  all topological groups to occur in the domain. It is only the
  ``nice'' topological groups, i.e.\ the automorphism groups of
  $\aleph _0$-categorical theories.) \end{conj}


\begin{conj} There is a 2-functor $\3F$ from the 2-category of
  $\aleph _0$-categorical theories and the 2-category of topological
  groups. \end{conj}







\printbibliography 



\end{document} 
%%% Local Variables:
%%% mode: latex
%%% TeX-master: t
%%% End:
